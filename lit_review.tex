\chapter{Literature Overview}
\label{cha:lit_overview}

\section{Theoretical Framework}

As the thesis is in nature a research and experimental project the theory will mostly be used as a reference in how internals of the Linux system work and how to approach analyzing them.

- documentation as most used source

\section{Key Studies}

- linux man pages \cite{kerrisk_mmap2_2024}

- open source documentation (such as the goblin ELF parser \cite{m4b_m4bgoblin_2024})

- publications referencing the used software \cite{drepper_how_2011}

- the Linux Programming interface by Michael Kerrisk \cite{kerrisk_linux_2010}

\section{Research Gaps}

- deep specifics of linking

- hard specifics of kernel memory safety features (e.g. ALSR)

\section{Key Concepts}

- Linux as a file-based operating system

- memory and virtual memory

- process and process memory mapping

- command line interface

- text/terminal user interface

- assembly and decompilation

- Linked and the dynamic linking process

- ELF libraries,executables and their internals

- foreign function interfaces

- rust-specific vocabulary
    
    - traits
    - modules and crates
    - cargo build system
    - panics
    
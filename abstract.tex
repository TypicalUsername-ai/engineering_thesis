\begin{abstract}

    As cybersecurity attacks are becoming more and more complicated and avoid detection by hiding closer and closer to the operating system, memory analysis of a running process is a logical next step in research and protection against such threats.
    The research goal of this project is to research information about the Linux systems file-based process interface, and find possible methods of extracting and parsing said information.
    The research is then applied in the creation of a program performing analysis and allowing for manual introspection into the memory of a running process.
    As the tool and the paper showcases even the fundamental analysis of the running process internals can provide deep and meaningful insights about the system of origin, dependencies, architecture, or even optimalizations and microcode used.
    This tool and research can be later used and extended to deepen and extend active threat protection and analysis at almost the kernel level.

\end{abstract}